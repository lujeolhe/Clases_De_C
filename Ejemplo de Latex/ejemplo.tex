%%%%%%%%%%%%%%%%%%%%%%%%%%%%%%%%%%%%%%%%%%%%%%%%%%
%Proyecto: Ejemplo de latex
%Colaboradores: Ljoh
%Fecha: 26 sep 2020
%%%%%%%%%%%%%%%%%%%%%%%%%%%%%%%%%%%%%%%%%%%%%%%%%%%

%================================================================
%				Tipo de documeto
%================================================================
\documentclass[12pt]{article}
%10pt,12pt twocolumns
%================================================================
%					Preambulo
%================================================================
\usepackage{lmodern}
\usepackage[utf8]{inputenc}
\usepackage[T1]{fontenc}
\usepackage[spanish,activeacute]{babel}

\usepackage{multicol}
\usepackage{enumerate}
\usepackage{enumitem}
\usepackage{booktabs}
\usepackage{tabularx, makecell}

\usepackage{mathtools}
\usepackage{amssymb, amsmath, amsbsy} % simbolitos
%\usepackage{upgreek} % para poner letras griegas sin cursiva
\usepackage{cancel} % para tachar
\usepackage{mathdots} % para el comando \iddots
\usepackage{mathrsfs} % para formato de letra
\usepackage{stackrel} % para el comando \stackbin
\usepackage{multirow} % para las tablas
%\usepackage[fleqn]{amsmath}
\usepackage{nccmath}
\usepackage{setspace}
%Circuitos
%\usepackage{tikz}
%\usepackage{circuitickz}


\usepackage{wasysym}

\usepackage[usenames]{color}
%================================================================
%					Comandos
%================================================================
\definecolor{ColorRespuesta}{RGB}{12,79,182}
\newcommand{\Respuesta}[1]{\textcolor{ColorRespuesta}{#1}}
\newcommand{\derivada}[2]{\displaystyle{\frac{d#1}{d#2}}}
\newcommand{\e}[1]{e^{#1}}
\newcommand{\integral}[4]{\displaystyle{\int_{#1}^{#2}{#3}{#4}}}
\newcommand{\escribir}[1]{\singlespacing#1\singlespacing}
\newcommand{\fraccion}[2]{\displaystyle{\frac{#1}{#2}}}
%================================================================
%					Margenes
%================================================================
\setlength{\textwidth}{170mm}%Ancho de Texto
\setlength{\textheight}{230mm}%Largo del Texto
\setlength{\oddsidemargin}{-5mm}%Margen de pagunas impares
\setlength{\evensidemargin}{5mm}%Margen de páginas pares
								%-para documentos tipo book-
\setlength{\topmargin}{-20mm}%Margeb Superior

%================================================================
%				  Datos del Autor
%================================================================
\title{Circuritos}
\author{Luis Pablo}
%\date{\today}
%================================================================
%					Documento
%================================================================


\begin{document}

\maketitle

	\escribir{Analisis del Circuito RC:}
	$V(t)=\fraccion{A\sin(wt)}{R_1}$\\

	$V_{C_1}=\fraccion{q(t)}{C_1} $\\

	$V_{R_1}=IR_1=\derivada{q(t)}{t}R_1$\\

	$V_{R_1}=IR_1=\derivada{q(t)}{t}R_1$\\

	\escribir{Comienza la solucion de la ecuacion diferecial para encontrar q(t)}
	$\derivada{q(t)}{t}+\fraccion{q(t)}{R_1C_1}=\fraccion{Asin(wt)}{R_1}$\\

	\escribir{Esto es para encontrar la ecuacion homogenea=0}
	$\derivada{q(t)}{t}+\fraccion{q(t)}{R_1C_1}=0$\\

	$\derivada{q(t)}{t}=-\fraccion{q(t)}{R_1C_1}$\\

	$\displaystyle\int{\fraccion{1}{q(t)}}{dq(t)}=\integral{0}{t}{\fraccion{1}{R_1C_1}}{dt}$\\

	$\ln (q(t)c)=-\fraccion{t}{R_1C_1}$\\

	\escribir{Esto se eleva a euler e}
	$q_{h}(t)=\fraccion{\e{\frac{-t}{R_1C_1}}}{c}$\\

	\escribir{Esto es para encontrar la ecuacion particular}
	$q_{p}(t)=m\sin(wt)+k\cos(wt)$\\

	$q_{p}(t)'=wm\cos(wt)-wk\sen(wt)$\\

	\escribir{Sustituimos en la ecuacion principal}
	$wm\cos(wt)-wk\sen(wt)+\fraccion{m\sin(wt)+k\cos(wt)}{R_1C_1}=\fraccion{A\sin(wt)}{R_1}$\\

	$wm\cos(wt)+\fraccion{k\cos(wt)}{R_1C_1}=0$\\

	$-wk\sen(wt)+\fraccion{m\sin(wt)}{R_1C_1}=\fraccion{A\sin(wt)}{R_1}$\\
	
	\escribir{Ahora se despejamos k con respecto de m}
	$wm\cos(wt)+\fraccion{k\cos(wt)}{R_1C_1}=0$\\

	$\fraccion{k}{R_1C_1}=-wm$\\

	$k=-wmR_1C_1$\\ 

	\escribir{Ahora se despejamos k y usamos el valor de m que acabamos de descubrir}
	$-wk\sen(wt)+\fraccion{m\sin(wt)}{R_1C_1}=\fraccion{A\sin(wt)}{R_1}$\\

	$\fraccion{m}{R_1C_1}-kw=\fraccion{A}{R_1}$\\
	\escribir{Sustituimos la k por lo que encontramos con anterioridad}
	$\fraccion{m}{R_1C_1}+mw^2R_1C_1=\fraccion{A}{R_1}$\\

	$m(\fraccion{1}{R_1C_1}+w^2R_1C_1)=\fraccion{A}{R_1}$\\

	$m=\fraccion{A}{(\fraccion{1+w^2R_1^2C_1^2}{R_1C_1})R_1}$\\

	$m=\fraccion{AC_1}{1+w^2R_1^2C_1^2}$\\
	
	\escribir{Ahora sustituimos m en la formula de k}
	$k=-\fraccion{AwC_1^2R_1}{1+w^2R_1^2C_1^2}$\\

	\escribir{Escribimos ahora la formula particular de q(t)}
	$q_p(t)=\fraccion{AC_1}{1+w^2R_1^2C_1^2}\sin(wt)-\fraccion{AwC_1^2R_1}{1+w^2R_1^2C_1^2}\cos(wt)$\\

	\escribir{Escribimos ahora la formula Equivalente de q(t)}
	$q_t(t)=a\e{\frac{-t}{R_1C_1}}+\fraccion{AC_1}{1+w^2R_1^2C_1^2}\sin(wt)-\fraccion{AwC_1^2R_1}{1+w^2R_1^2C_1^2}\cos(wt)$\\
	
	\escribir{Ahora encontraremos el valor de a}
	\escribir{Si t=0, entocnes q(0)=0}
	$0=a-\fraccion{AwC_1^2R_1}{1+w^2R_1^2C_1^2}$\\
	
	$a=\fraccion{AwC_1^2R_1}{1+w^2R_1^2C_1^2}$\\

	\escribir{Por lo que la solución nos queda}
	$q(t)=\fraccion{AwC_1^2R_1}{1+w^2R_1^2C_1^2}\e{\frac{-t}{R_1C_1}}+\fraccion{AC_1}{1+w^2R_1^2C_1^2}\sin(wt)-\fraccion{AwC_1^2R_1}{1+w^2R_1^2C_1^2}\cos(wt)$\\

	\escribir{Factorizamos:}
	$q(t)=\fraccion{AC_1}{1+w^2R_1^2C_1^2}(wR_1C_1\e{\frac{-t}{R_1C_1}}+\sin(wt)-wR_1C_1\cos(wt))$\\

	\escribir{Derivado:}

	$q(t)'=\fraccion{AC_1}{1+w^2R_1^2C_1^2}(-w\e{\frac{-t}{R_1C_1}}+w\cos(wt)+w^2R_1C_1\sin(wt))$\\

		
	\escribir{Sustituimos los valores de $q(t)$ y $q(t)'$  en la formalas de $V_{C_1}$ y $V_{R_1}$ }

	$V_{C_1}=\fraccion{q(t)}{C_1} $\\

	$V_{R_1}=IR_1=\derivada{q(t)}{t}R_1$\\

	\escribir{Quedan así:}
	$V_{C_1}=\fraccion{AC_1^2}{1+w^2R_1^2C_1^2}(wR_1C_1\e{\frac{-t}{R_1C_1}}+\sin(wt)-wR_1C_1\cos(wt))$\\


	$V_{R_1}=\fraccion{AC_1R_1}{1+w^2R_1^2C_1^2}(-w\e{\frac{-t}{R_1C_1}}+w\cos(wt)+w^2R_1C_1\sin(wt))$\\

    %Casos de uso de Inicio de Sesión

\end{document}
